\input {preamble}\begin {document}\fancyhead {}\fancyfoot {}
\par \noindent \textbf {\textnumero 1}\par \begin {otherlanguage*}{english}\makeatletter \textbf {Brak \,Ivan\unhbox \voidb@x \nobreak \,V.\unskip {}, \ignorespaces Leladze \,Konstantin\unhbox \voidb@x \nobreak \,G.\unskip {}, \ignorespaces Kiselev \,Gleb\unhbox \voidb@x \nobreak \,A.}\makeatother \par \textit {Reconstructing the Volumetric Geometry of the Cerebral Cortex Based on Magnetic Resonance Imaging}\par This review article focuses on the analysis of approaches for reconstructing the volumetric geometry of the cerebral cortex and white matter tracts based on magnetic resonance imaging (MRI) data. Particular attention is given to voxel-based morphometry (VBM) and surface-based morphometry (SBM) — two key methodologies for the quantitative analysis of brain structure. The algorithmic principles of each method are considered, along with their advantages, limitations, and metric characteristics, including cortical thickness, surface area, and gyrification index. \par A separate focus is placed on comparing the sensitivity and accuracy of VBM and SBM in identifying morphological changes related to cognitive and neuropsychiatric traits. Additionally, the article discusses the promising tensor-based surface morphometry (TBSM), which is based on differential geometry methods and provides high spatial resolution without topological distortion. \par This work emphasizes the importance of accurate and reproducible structural brain reconstruction in the context of biomedical technology and neuroscience development. \par \keywordsname : brain morphometry, SBM, VBM, cortical thickness, MRI, anatomical reconstruction, neuroimaging reproducibility \end {otherlanguage*}\par \par \medskip 
\par \begin {otherlanguage*}{russian}\makeatletter \textbf {Брак ~И.\unhbox \voidb@x \nobreak \,В.\unskip {}, \ignorespaces Киселёв ~Г.\unhbox \voidb@x \nobreak \,А.\unskip {}, \ignorespaces Леладзе ~К.\unhbox \voidb@x \nobreak \,Г.}\makeatother \par \textit {Восстановление объемной геометрии коры головного мозга по результатам магнитно резонансной томографии}\par Настоящая обзорная статья посвящена анализу подходов к восстановлению объемной геометрии коры головного мозга и проводящих путей на основе данных магнитно-резонансной томографии. Основное внимание уделено воксельно-ориентированной (VBM) и поверхностно-ориентированной (SBM) морфометрии — двум ключевым направлениям в количественном анализе структуры мозга. Рассматриваются алгоритмические принципы каждого подхода, их достоинства, ограничения, а также метрические характеристики, включая толщину коры, площадь поверхности и индекс гирификации. \par Отдельный акцент сделан на сравнении чувствительности и точности VBM и SBM в выявлении морфологических изменений, связанных с когнитивными и нейропсихическими особенностями. Также рассматривается перспективный тензорно-ориентированный подход (TBSM), опирающийся на методы дифференциальной геометрии и обеспечивающий высокую разрешающую способность без искажения топологии. \par Работа подчёркивает значимость точной и воспроизводимой структурной реконструкции мозга в контексте развития биомедицинских технологий и нейронауки. \par \keywordsname : морфометрия мозга, SBM, VBM, толщина коры, МРТ, анатомическая реконструкция, воспроизводимость нейровизуализации \par \end {otherlanguage*}\par \vspace \bigskipamount 
\par \begin {otherlanguage*}{english}\makeatletter \textbf {Брак ~И.\unhbox \voidb@x \nobreak \,В.\unskip {}, \ignorespaces Киселёв ~Г.\unhbox \voidb@x \nobreak \,А.\unskip {}, \ignorespaces Леладзе ~К.\unhbox \voidb@x \nobreak \,Г.}\makeatother \par \textit {Восстановление объемной геометрии коры головного мозга по результатам магнитно резонансной томографии}\par Настоящая обзорная статья посвящена анализу подходов к восстановлению объемной геометрии коры головного мозга и проводящих путей на основе данных магнитно-резонансной томографии. Основное внимание уделено воксельно-ориентированной (VBM) и поверхностно-ориентированной (SBM) морфометрии — двум ключевым направлениям в количественном анализе структуры мозга. Рассматриваются алгоритмические принципы каждого подхода, их достоинства, ограничения, а также метрические характеристики, включая толщину коры, площадь поверхности и индекс гирификации. \par Отдельный акцент сделан на сравнении чувствительности и точности VBM и SBM в выявлении морфологических изменений, связанных с когнитивными и нейропсихическими особенностями. Также рассматривается перспективный тензорно-ориентированный подход (TBSM), опирающийся на методы дифференциальной геометрии и обеспечивающий высокую разрешающую способность без искажения топологии. \par Работа подчёркивает значимость точной и воспроизводимой структурной реконструкции мозга в контексте развития биомедицинских технологий и нейронауки. \par \keywordsname : морфометрия мозга, SBM, VBM, толщина коры, МРТ, анатомическая реконструкция, воспроизводимость нейровизуализации \par \end {otherlanguage*}\par \vspace \bigskipamount 
